%{{{ Formatierung

\documentclass[a4paper,12pt]{article}

\usepackage{physics_notetaking}

%%% dark red
%\definecolor{bg}{RGB}{60,47,47}
%\definecolor{fg}{RGB}{255,244,230}
%%% space grey
%\definecolor{bg}{RGB}{46,52,64}
%\definecolor{fg}{RGB}{216,222,233}
%%% purple
%\definecolor{bg}{RGB}{69,0,128}
%\definecolor{fg}{RGB}{237,237,222}
%\pagecolor{bg}
%\color{fg}

\newcommand{\td}{\,\text{d}}
\newcommand{\RN}[1]{\uppercase\expandafter{\romannumeral#1}}
\newcommand{\zz}{\mathrm{Z\kern-.3em\raise-0.5ex\hbox{Z} }}
\newcommand{\id}{1\kern-.258em1}

\newcommand\inlineeqno{\stepcounter{equation}\ {(\theequation)}}
\newcommand\inlineeqnoa{(\theequation.\text{a})}
\newcommand\inlineeqnob{(\theequation.\text{b})}
\newcommand\inlineeqnoc{(\theequation.\text{c})}

\newcommand\inlineeqnowo{\stepcounter{equation}\ {(\theequation)}}
\newcommand\inlineeqnowoa{\theequation.\text{a}}
\newcommand\inlineeqnowob{\theequation.\text{b}}
\newcommand\inlineeqnowoc{\theequation.\text{c}}

\renewcommand{\refname}{Source}
\renewcommand{\sfdefault}{phv}
%\renewcommand*\contentsname{Contents}

\pagestyle{fancy}

\sloppy

\numberwithin{equation}{section}

%}}}

\begin{document}

%{{{ Titelseite

\title{5 (1.\ Halbtag) $|$ Operationsverstärker}
\author{Angelo Brade, Jonas Wortmann}
\maketitle
\pagenumbering{gobble}

%}}}

\newpage

%{{{ Inhaltsverzeichnis

\fancyhead[L]{\thepage}
\fancyfoot[C]{}
\pagenumbering{arabic}

\tableofcontents

%}}}

\newpage

%{{{

\fancyhead[R]{\leftmark\\\rightmark}

\newpage
\section{Einleitung}

\newpage
\section{Theorie}

\newpage
\section{Voraufgaben}
\subsection{A}
Es gilt die Formel
\begin{align} 
        \dfrac{1}{v}&=\dfrac{1}{v_0}+k&v&=\dfrac{1}{\tfrac{1}{v_0}+k}
.\end{align} 
Für die Werte $k=0.1$, $v_0=10^4$ und $v_0=10^5$ ergibt sich
\begin{align} 
        v_1&\approx 9.990 & v_2&\approx 9.999
.\end{align} 
Mit der Näherung von $v=\tfrac{1}{k}$ ergibt sich
\begin{align} 
        v_\text{Näh}&=10
.\end{align} 
Die Abweichung von $v_1$ und $v_2$ von $v_\text{Näh}$ liegen jeweils bei $0.001\%$ und $0.0001\%$.

\subsection{B}
Es gilt
\begin{align} 
        && U_x &= U_\text{in}-kU_\text{out} && \\
        \Leftrightarrow && &= U_\text{in}-kv_0U_x &&\nonumber \\
        \Leftrightarrow && &= \dfrac{U_\text{in}}{1+v_0k}. &&
\end{align} 
Für $k=0.1$, $v_0=10^5$ und $U_\text{in}=\SI{1}{V}$ ist
\begin{align} 
        U_x\approx \SI{0.0001}{V}
.\end{align} 


\newpage
\section{Auswertung}

\clearpage
\listoffigures
\listoftables
%\bibliographystyle{plain}
%\bibliography{refs}

%}}}

\end{document}
