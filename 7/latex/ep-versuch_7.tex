\documentclass[10pt]{article}
\usepackage[a4paper, left=1.5cm, right=1.5cm, top=3.5cm]{geometry}
\usepackage[ngerman]{babel}
\usepackage[]{graphicx}
\usepackage{multicol}
\usepackage{amssymb}
\usepackage{inputenc}
\usepackage{breqn}
\usepackage{titlesec}
\usepackage{wrapfig}
\usepackage{blindtext}
\usepackage{lipsum}
\usepackage{caption}
\usepackage{listings}
\usepackage{fancyhdr}
\usepackage{nopageno}
\usepackage{authblk}
\usepackage{amsmath}
\usepackage{mathtools}
\usepackage{bm}
\usepackage[ISO]{diffcoeff}
\usepackage{xcolor}
\usepackage{csquotes}
\usepackage{siunitx}
\usepackage{circuitikz}
\fancyhf[]{}

\newenvironment{Figure}
  {\par\medskip\noindent\minipage{\linewidth}}
  {\endminipage\par\medskip}

\begin{titlepage}
    \title{Elektronikpraktikum -- Versuch 7}
    \author[1]{Jonas Wortman\thanks{s02jwort@uni-bonn.de}}
    \author[1]{Angelo V. Brade\thanks{s72abrad@uni-bonn.de}}
    \affil[1]{Rheinische Friedrich-Wilhelms-Universität Bonn}
    \date{\today}
\end{titlepage}

\begin{document}
\pagenumbering{gobble}
\maketitle
\newpage

\tableofcontents
\newpage

\pagenumbering{arabic}

\pagestyle{fancy}
\fancyhead[R]{\thepage}
\fancyhead[L]{\leftmark}


\begin{multicols}{2}
	\section{Einleitung}

	\section{Theorie}
	\subsection{Boolische Algebra und Schaltfunktionen}
	Bei digitalen Schalftelementen gibt es im allgmeinen mehrere Eingänge und einen Ausgang, wobei alle Signal als 0 oder 1 interpretiert werden. Das Verhalten lässt sich mit Schaltfunktionen beschreiben, die von der Menge der Eingangsvariablen auf die Menge der Ausgangsvariable abbildet. Um diese Funktionen darzustellen werden Funktionstafeln verwendet, die alle Kombinationen an Eingängen, sowie die zugeordneten Ausgang angeben.

	Für eine Eingangsvariable sind die folgenden operationen möglich:
	\begin{align*}
		 & \textbf{Identität}                                  & p(x) = x       \\
		 & \textbf{Komplement} \text{ oder } \textbf{Negation} & p(x) = \bar{x} \\
		 & \text{sowie konstant 1}                             & p(x) = 0       \\
		 & \text{und konstant 0}                               & p(x) = 1
	\end{align*}

	Für Funktionen mit zwei Eingansvariablen, sog. elementare Funktionen, sind diese möglich:
	\begin{center}
		\begin{tabular}{|c|c|c|c|}
			\hline
			      &       & \textbf{Konjunktion} & \textbf{Disjunktion} \\
			$x_1$ & $x_2$ & \textbf{UND}         & \textbf{ODER}        \\
			      &       & $x_1 \cdot x_2$      & $x_1 + x_2$          \\
			\hline
			0     & 0     & 0                    & 0                    \\
			0     & 1     & 0                    & 1                    \\
			1     & 0     & 0                    & 1                    \\
			1     & 1     & 1                    & 1                    \\
			\hline
		\end{tabular}
		\captionof{table}{Elementare Funktionen}
	\end{center}

	Nun lässt sich jede boolesche Funktion F als Summe von Mintermen, die sog. disjunktive Normalform, oder als Produkt von Maxtermen, die sog. konjunktive Normalform, ausdrücken. Hierbei ist eine Minterm $m$ eine boolesches Produkt aus jeder Eingangsvariable oder ihrem Kompliment, wobei diese jeweils nur einmal auftreten. Genauso ist ein Maxterm $M$ eine boolesche Summer aus jeder Eingangsvariable oder ihrem Kompliment, wobei diese jeweils nur einmal auftreten.
	Hier sind gebräuchliche Schalftunktionen dargestellt:
	\begin{Figure}
		\centering
		\includegraphics[width=1\textwidth]{schaltfunktionen.png}
		\captionof{figure}{Gebräuchliche Schaltfunktionen}
	\end{Figure}
	\subsection{Flip-Flops}
	Die einfachste möglichkeit ein Signal zu speicher, ist mithilfe eines Flip-Flops.
	\begin{Figure}
		\centering
		\includegraphics[width=1\textwidth]{fliflop.png}
		\captionof{figure}{Flip-Flop}
	\end{Figure}
	Ist $a=b=1$, so ist das signal gespeichert und es lässt sich durch belegen mit einer 0 löschen. Hierbei ist dann $Q_2=\overline{Q_1}$ gespeichert.
	\subsection{Schrieberegister und Zähler}
	Nun kann man mithilfe von FFs eine Schreibregister bauen, dass mit jedem Takt den Inhalt des $i$-ten FFs in das $(i+1)$-te FF schreibt. Neben dem gibt es dann noch den Dualzähler. Er erzeugt in ansteigender Reinfolge mit jedem Takt Dualzahlen.
	\subsection{Aufbau von elektronischen Logikschaltungen}
	Um nun Logikschaltungen zu konstruieren, wird eine Minimalspannung, ab der das Signal als 1 interpretiert wird, und eine Maximalspannung, unter der das Signal als 0 interpretiert wird, definiert. Diese Schwellspannungen lassen sich in Übertragungskennlinien darstellen, hier in Abb. \ref{fig:kennlinie}.
	\begin{Figure}
		\centering
		\includegraphics[width=1\textwidth]{übetragungskennlinie.png}
		\captionof{figure}{Übertragungskennlinie eines Inverters}
		\label{fig:kennlinie}
	\end{Figure}
	\begin{Figure}
		\centering
		\includegraphics[width=1\textwidth]{inverter.png}
		\captionof{figure}{Schaltung eines Inverters}
		\label{fig:inverter}
	\end{Figure}
	Z.B ein Inverter kann nun mit Transistoren realisiert werden. Hier in Abb. \ref{fig:inverter} dargestellt. Diese Schaltung kann mit z.B. Dioden erweitert werden, um mehr Eingänge zu erhalten. Dies wird allerding heutzutage nicht mehr mit Diodon, sondern direkt mit Transistoren gemacht, da diese deutlich kleiner hergestellt werden können. So kann auch ein CMOS-Gatter realisiert werden, welches allein aus Transistoren besteht und somit extrem klein gebaut werden kann. Dies ist hier in Abb. \ref{fig:gatter} dargestellt.
	\begin{Figure}
		\centering
		\includegraphics[width=1\textwidth]{cmos-gatter.png}
		\captionof{figure}{Schaltung eines CMOS-Gatter}
		\label{fig:gatter}
	\end{Figure}
	\section{Voraufgaben}
	\subsection*{A}
	Wenn man $n$ Eingangsvariablen hat, gibt es ohne Redundanzen $2^{2n}$ Schaltfunktionen.
	\subsection*{B}
	Es werden folgende Ausdrücke mit einer Funktionstafel überprüft.
	\begin{align*}
		a + 1 & = 1 \quad & a \cdot 1 & = a \quad & a + \bar{a}     & = 1 \\
		a + 0 & = a \quad & a \cdot 0 & = 0 \quad & a \cdot \bar{a} & = 0
	\end{align*}
	Man erkennt in Tab. \ref{tab:ftafel}, dass alle Ausrücke korrekt sind.

	\begin{center}
		\begin{tabular}{|c|c|c|c|}
			\hline
			\textbf{Funktion $f$} & $a$ & $\bar{a}$ & $f(a, \bar{a})$ \\
			\hline
			$a + 1$               & 0   & 1         & 1               \\
			                      & 1   & 0         & 1               \\
			\hline
			$a + 0$               & 0   & 1         & 0               \\
			                      & 1   & 0         & 1               \\
			\hline
			$a \cdot 1$           & 0   & 1         & 0               \\
			                      & 1   & 0         & 1               \\
			\hline
			$a \cdot 0$           & 0   & 1         & 0               \\
			                      & 1   & 0         & 0               \\
			\hline
			$a + \bar{a}$         & 0   & 1         & 1               \\
			                      & 1   & 0         & 1               \\
			\hline
			$a \cdot \bar{a}$     & 0   & 1         & 0               \\
			                      & 1   & 0         & 0               \\
			\hline
		\end{tabular}
		\captionof{table}{Funktionstafel}
		\label{tab:ftafel}
	\end{center}
	\subsection*{C}
	Nun wird das Distributivgesetz und das Gesetz von DeMorgan mithilfe von Funktionstafeln bestätigt. Diese sind in Tab. \ref{tab:dis} und \ref{tab:morg} dargestellt.
	\begin{center}
		\begin{tabular}{|c|c|c|c|c|c|c|c|c|}
			\hline
			$a$ & $b$ & $c$ & $a + b$ & $(a + b) \cdot c$ & $a \cdot c$ & $b \cdot c$ & $a \cdot c + b \cdot c$ \\
			\hline
			0   & 0   & 0   & 0       & 0                 & 0           & 0           & 0                       \\
			0   & 0   & 1   & 0       & 0                 & 0           & 0           & 0                       \\
			0   & 1   & 0   & 1       & 0                 & 0           & 0           & 0                       \\
			0   & 1   & 1   & 1       & 1                 & 0           & 1           & 1                       \\
			1   & 0   & 0   & 1       & 0                 & 0           & 0           & 0                       \\
			1   & 0   & 1   & 1       & 1                 & 1           & 0           & 1                       \\
			1   & 1   & 0   & 1       & 0                 & 0           & 0           & 0                       \\
			1   & 1   & 1   & 1       & 1                 & 1           & 1           & 1                       \\
			\hline
		\end{tabular}

		\captionof{table}{Funktionstafel für das Distributivgesetz}
		\label{tab:dis}
	\end{center}
	\begin{center}
		\begin{tabular}{|c|c|c|c|c|c|c|c|c|c|c|}
			\hline
			$a$ & $b$ & $\bar{a}$ & $\bar{b}$ & $a \cdot b$ & $\overline{a\cdot b}$ & $\bar{a} + \bar{b}$ & $a + b$ & $\bar{a + b}$ & $\overline{a} \cdot \bar{b}$ \\
			\hline
			0   & 0   & 1         & 1         & 0           & 0                     & 1                   & 0       & 1             & 1                            \\
			0   & 1   & 1         & 0         & 0           & 1                     & 0                   & 1       & 0             & 1                            \\
			1   & 0   & 0         & 1         & 0           & 0                     & 1                   & 1       & 1             & 0                            \\
			1   & 1   & 0         & 0         & 1           & 0                     & 0                   & 1       & 0             & 0                            \\
			\hline
		\end{tabular}
		\captionof{table}{Funktionstafel für das DeMorgan-Gesetz}
		\label{tab:morg}
	\end{center}
	\subsection*{D}
	\begin{align}
		f(a, b) & = (a + b) \cdot (\overline{a\cdot b})
		        & =
	\end{align}
	Der ausruck folgt der Tabelle, allerdings beschriebt $\overline{(a\cdot b)}$ nicht die Tabelle.
	\subsection*{E}
	Da wir für jede Variable entweder die normale Form oder die negierte Form und somit zwei Zustände haben, und wir jede Kombination durchgehen wollen, kann man binär hochzählen, wobei dann 0 für normal und 1 für negiert steht.
	\begin{center}
		\begin{tabular}{|c|c|}
			\hline
			Binäre Zahl & entsprechender Minterm                               \\
			\hline
			000         & $a \cdot b \cdot c$                                  \\
			001         & $a \cdot b \cdot \overline{c}$                       \\
			010         & $a \cdot \overline{b} \cdot c$                       \\
			011         & $a \cdot \overline{b} \cdot \overline{c}$            \\
			100         & $\overline{a} \cdot b \cdot c$                       \\
			101         & $\overline{a} \cdot b \cdot \overline{c}$            \\
			110         & $\overline{a} \cdot \overline{b} \cdot c$            \\
			111         & $\overline{a} \cdot \overline{b} \cdot \overline{c}$ \\
			\hline
		\end{tabular}
		\captionof{table}{Mintermbestimmung mithilfe von binärem Zählen.}
	\end{center}
	\subsection*{F}
	Sobald $a$ und $b$, 1 sind, sind die Ausgänge nicht mehr eindeutig. Für sonnstige Zustände sind die Ausgänge eindeutig.
	\begin{center}
		\begin{tabular}{|c|c|c|c|}
			\hline
			$a$ & $b$ & $Q_1$    & $Q_2$    \\
			\hline
			0   & 0   & 1        & 1        \\
			0   & 1   & 1        & 0        \\
			1   & 0   & 0        & 1        \\
			1   & 1   & 0 oder 1 & 0 oder 1 \\
			\hline
		\end{tabular}
		\captionof{table}{Funktionstafel von einem Flip Flop}
		\label{tab:fff}
	\end{center}
	\subsection*{G}
	Um einen 4-Bit-Schreibergeister zu konstruieren, müssen vier D-FF hinteinander geschalten werden, welche snychron getrakted werden. Dieser ist in Abb. \ref{fig:schreibregister} dargestellt.
	\begin{Figure}
		\centering
		\includegraphics[width=1\textwidth]{circuit-20240908-2016.png}
		\captionof{figure}{4-Bit Schreibregister}
		\label{fig:schreibregister}
	\end{Figure}
	\subsection*{H}
	Ein paralleles Schieberegister wird nach Abb. \ref{fig:paraschreibregister} konstruiert.
	\begin{Figure}
		\centering
		\includegraphics[width=1\textwidth]{parallel-schieberegister.png}
		\captionof{figure}{parallel 4-Bit Schreibregister}
		\label{fig:paraschreibregister}
	\end{Figure}
	\subsection*{I}
	Ein Dualzähler wird nach Abb. \ref{fig:dualzähler} konstruiert.
	\begin{Figure}
		\centering
		\includegraphics[width=1\textwidth]{4bit_dualzähler.png}
		\captionof{figure}{parallel 4-Bit Dualzähler}
		\label{fig:dualzähler}
	\end{Figure}
	\subsection*{J}
	Die Funktion der Schaltung entspricht einer NOR-Schaltung. Die beiden Dioden dienen als Eingangssignal, das aktiviert ist, sobald einer der beiden Eingänge aktiv ist. Somit erhälten wir eine OR-Schaltung. Nun ist hinter den Dioden ein Invertierer geschaltet, welcher das Signal invertiert und so die Schaltung zu einer NOR-Schaltung vervollständigt.

	Da wir eine Basis-Spannung von ca. \SI{1.8}{V} haben und über Basis-Emitter $U_{BE}=\SI{0.7}{V}$ abfällt können wir mit
	\begin{align}
		I_B & =\frac{\SI{1.8}{V}-\SI{0.7}{V}}{\SI{1}{\kilo\Omega}}-\frac{\SI{0.7}{V}}{\SI{1}{\kilo\Omega}} \\
		    & =\SI{0.4}{\milli A}
	\end{align}
	zeigen, dass der Ausgangspegel korrekt ist.
	\subsection*{K}
	Es fließt Strom, aber es gibt probleme.

	Muss noch gemacht werden!!!!!!1
	\subsection*{L}
	Beim CMOS-Inverter fließt bei geringer anliegender Spannung Strom, sodass beim Ausgang Spannung anliegt. Bei hoher Spanung fließt kein Strom, sodass keine Ausgangsspannung anliegt.
	\subsection*{M}
	Die Zeichnung realisiert ein NAND-Gatter. Es ist immer logisch 1, außer beide Eingänge sind logisch 1.
	\subsection*{N}
	Ein Übetrag eines Halbaddierer ist ein XOR und die Summe eine AND.
	\subsection*{O}
	Die Funktionstafel eines Volladierers mit vorherigem Bit ist in Abb. \ref{tab:volladdierer} zu sehen.
	\begin{center}
		\begin{tabular}{|c|c|c|c|c|}
			\hline
			$a$ & $b$ & $S$ & Ü & Ü$_{i-1}$ \\
			\hline
			0   & 0   & 0   & 0 & 0         \\
			0   & 0   & 1   & 0 & 1         \\
			0   & 1   & 1   & 0 & 0         \\
			0   & 1   & 0   & 1 & 1         \\
			1   & 0   & 1   & 0 & 0         \\
			1   & 0   & 0   & 1 & 1         \\
			1   & 1   & 0   & 1 & 0         \\
			1   & 1   & 1   & 1 & 1         \\
			\hline
		\end{tabular}
		\captionof{table}{Funktionstafel eines Volladierers mit vorheriegem Bit}
		\label{tab:volladdierer}
	\end{center}

	\subsection*{P}
	In Abb. \ref{fig:volladdierer} ist ein Volladdierer dargestellt.
	\begin{Figure}
		\centering
		\includegraphics[width=1\textwidth]{volladdierer.png}
		\captionof{figure}{Volladdierer aus zwei Halbaddierern.}
		\label{fig:volladdierer}
	\end{Figure}

	\subsection*{Q}
	In Abb. \ref{fig:sadd} ist ein serielles Addierwerk dargestellt.
	\begin{Figure}
		\centering
		\includegraphics[width=0.8\textwidth]{serielles_addierwerk.png}
		\captionof{figure}{Serielles Addierwerk}
		\label{fig:sadd}
	\end{Figure}
	\section{Auswertung}
	\section{Fazit}
\end{multicols}
\end{document}
